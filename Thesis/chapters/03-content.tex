\chapter{Theoretical Shenanigans}

\begin{theorem}
When given a polygon P and a guard set G, a subset of G fully covers P if and only if it covers all witnesses in the shadow witness set of G.
\end{theorem}
\begin{proof}

\end{proof}

\begin{theorem}
When given a polygon P and a guard set G, the guard subsets of a light polygon form a clique in the 2-link-visibility graph of G and the cliques of all light polygons cover all edges of the graph.
\end{theorem}
\begin{proof}
    
\end{proof}

\begin{theorem}
For a polygon P and a guard set G, neither conflict-freeness of all shadow witnesses nor all light witnesses guarantees conflict-freeness within all AVPs.
\end{theorem}
\begin{proof}
To prove this theorem, we present a simple example in.
\end{proof}

\chapter{Chromatic Art Gallery Problem Formulations}

\section{MIP Formulation}

\section{SAT Formulation}

\section{CPSAT Formulation}

\chapter{Conflict-free Chromatic Art Gallery Problem Formulations}

\section{MIP Formulation}

\section{SAT Formulation}

\section{CPSAT Formulation}

\chapter{Implementation Details}

\section{Instance Processing}

\section{Initial Upper Bound}

\section{Clique Edge Covers}

\section{Lazy Constraints}