\chapter{Preliminaries}
\begin{definition}[Simple polygon]
When given a list of distinct points in the plane, the line segments between consecutive points in the list and the last and first point form the outer boundary of a simple polygon, as long as none of them cross each other. This outer boundary is a Jordan polygonal curve, and together with its interior face, it forms a simple polygon. A simple polygon with holes P, additionally contains one or more smaller simple polygons fully contained in its interior. The interior of these smaller polygons is then considered as the exterior of P.
\end{definition}

\begin{definition}[Visibility polygon]
For a simple polygon P (possibly with holes), two points p, q E P are mutually visible if the line segment between them is fully contained in P. Given a point $p\in P$, p and all points in P visible to it form a simple polygon. We call the polygon visibility polygon of p.
\end{definition}

\begin{definition}[Guard set]
For a simple polygon P (possibly with holes), a guard set G is a set of points in P s.t. all points in P are contained in at least one of the visibility polygons of G. We call the points of the guard set guards.
\end{definition}

\begin{definition}[Atomic visibility polygon (AVP)]
When given a guard set G of a polygon P, we can overlay the visibility polygons of G to form an arrangement. We call the faces of this arrangement atomic visibility polygon (AVP) and denote the set of faces as $\mathcal{F}$. By assigning to each AVP $f\in\mathcal{F}$ the guard subset $G_{f}$ that they were created from, we can form a partial order over the AVPs s.t. for $f,f'\in\mathcal{F}$ $f\prec f'$ if and only if $G_{f}\subsetG_{f'}$. For an AVP $f\in\mathcal{F}$ we call it shadow (light) whenever it is minimal (maximal) within that order.
\end{definition}

\begin{definition}[Shadow witness set]
When given a guard set G of a polygon P, the shadow witness set contains a witness 
\end{definition}

\begin{definition}[2-link-visibility graph]
    
\end{definition}

\begin{definition}[Covering graph]
    
\end{definition}

\begin{definition}[Chromatic Art Gallery Problem]
    
\end{definition}

\begin{definition}[Conflict-free Chromatic Art Gallery Problem]
    
\end{definition}