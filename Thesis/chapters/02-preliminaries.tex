\chapter{Preliminaries}\label{ch:preliminaries}
In this chapter, we will formally define a few terms used throughout the following chapters. We use the same terminology as Zambon et al. in their paper~\cite{zambon2014exact}.
First of all, our input is given as a simple polygon possibly with holes.

\begin{definition}[Simple polygon]
When given a list of distinct points in the plane, the line segments between consecutive points in the list and the last and first point form the outer boundary of a simple polygon as long as none cross each other. This outer boundary is a Jordan polygonal curve, and together with its interior face, it forms a simple polygon. A simple polygon with holes $P$ additionally contains one or more smaller simple polygons fully contained in its interior. The interior of these smaller polygons is then considered as the exterior of $P$.
\end{definition}

Within this polygon, we use a set of guards to cover it using their visibility polygons and witnesses to check if certain areas are covered.

\begin{definition}[Visibility polygon]
For a simple polygon $P$ (possibly with holes), two points $p,q\in P$ are mutually visible if the line segment between them is fully contained in $P$. We say that $p$ covers $q$ and $q$ covers $p$. Given a point $p\in P$, $p$ and all points in $P$ visible to it form a simple polygon. We call this polygon visibility polygon of $p$.
\end{definition}

\begin{definition}[Guards and witnesses]
A guard is a point inside a simple polygon $P$ (possibly with holes) that covers the area within its visibility polygon. A witness is a point inside $P$ that must be covered by at least one guard.
\end{definition}

\begin{definition}[Guard set]
For a simple polygon $P$ (possibly with holes), a guard set $G$ is a set of guards in $P$ s.t. all points in $P$ are contained in at least one of the visibility polygons of $G$.
\end{definition}

Next, we will divide our polygon into certain areas called AVPs. This helps us define a finite set of witnesses that ensures coverage of the entire polygon.

\begin{definition}[Atomic visibility polygon (AVP)]
When given a guard set $G$ of a polygon $P$, we can overlay the visibility polygons of $G$ to form an arrangement. We call the faces of this arrangement atomic visibility polygon (AVP) and denote the set of faces as $\mathcal{F}$. When a guard of $G$ covers a point within an AVP $f$, it automatically also covers the entirety of $f$, hence the name atomic. By assigning to each AVP $f\in\mathcal{F}$ the guard subset $G_{f}$ that they were created from, we can form a partial order over the AVPs s.t. for $f,f'\in\mathcal{F}$ $f\prec f'$ if and only if $G_{f}\subset G_{f'}$. For an AVP $f\in\mathcal{F}$ we call it shadow (light) whenever it is minimal (maximal) within that order.
\end{definition}

\begin{definition}[Witness set]
When given a guard set $G$ of a polygon $P$, a witness set $W$ is a set of points inside $P$ that $G$ must cover. $W$ is called a shadow witness set when it consists of a point within each shadow AVP of $G$. We call these points shadow witnesses.
\end{definition}

The following two definitions allow us to reduce our problem definitions to graph problems.

\begin{definition}[2-link-visibility graph]
For a guard set $G$ of a polygon $P$, the 2-link-visibility graph $G_{vis}$ has a vertex for each guard and an edge between two guard vertices whenever their visibility polygons intersect.
\end{definition}

\begin{definition}[Covering graph]
For a guard set $G$ and a witness set $W$ of a polygon $P$, the covering graph $G_{cov}$ has a vertex for each guard in $G$ and each witness in $W$ and an edge between a guard and a witness whenever the witness is contained in the visibility region of that guard.
\end{definition}

Lastly, we will give a formal definition of the Chromatic Art Gallery Problem and Conflict-free Chromatic Art Gallery Problem using the before-established terminology.

\begin{definition}[Chromatic Art Gallery Problem (CAGP)]
For a guard set $G$ and a witness set $W$ of a polygon $P$, the Chromatic Art Gallery Problem looks for minimum proper coloring among all subgraphs $G_{sub}$ of $G_{vis}$ s.t. each witness vertex in $G_{cov}$ has a guard vertex in its neighborhood that is contained in $G_{sub}$.
\end{definition}

\begin{definition}[Conflict-free Chromatic Art Gallery Problem (CFCAGP)]
For a guard set $G$ and a witness set $W$ of a polygon $P$, the Conflict-free Chromatic Art Gallery Problem looks for a minimum coloring among all subgraphs $G_{sub}$ of $G_{cov}$ s.t. among the guard vertices in the neighborhood of each witness vertex, there exists at least one that has a unique color.
\end{definition}