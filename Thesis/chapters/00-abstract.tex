\chapter*{Abstract}
% In this thesis, we explore optimally solving the Chromatic Art Gallery Problem on vertex guards. This problem has previously been considered by Zambon et al. using a MIP solver. We reproduce these results and propose a different approach using SAT and CP-SAT solver. The usage of SAT solvers turned out to be much better allowing to solve random simple polygons without holes of up to 30,000 vertices as well as better performance on random simple polygons with holes. Additionally, we provide a MIP and SAT formulation for a variation of the problem called the Conflict-free Chromatic Art Gallery problem. This problem has never practically been considered before and turned out to be much harder than the original Chromatic Art Gallery Problem. Still, we managed to solve random simple polygons without holes of up to 2500 vertices as well as simple polygons with holes of up to TODO vertices.
Achieving comprehensive signal coverage is vital for indoor wireless and infrared communication. However, overlapping signals can lead to interference, necessitating the use of distinct frequencies in overlapping zones. With a limited available spectrum, optimizing frequency allocation becomes crucial to minimize interference while maintaining complete coverage. These challenges are encapsulated in the Chromatic Art Gallery Problem and the Conflict-Free Chromatic Art Gallery Problem, with solutions benefiting various applications including telecommunications, smart environments, and Internet of Things (IoT) deployments.

In this thesis, we investigate optimally solving the Chromatic Art Gallery Problem with vertex guards. The goal is to find a set of guards that covers the entirety of a (simple) polygon whilst requiring a minimum number of colors such that each guard gets assigned a color and no two guards of the same color have overlapping coverage areas.

Building upon previous work by Zambon et al.~\cite{zambon2014exact}, who used MIP solvers and tested random simple polygon instances without holes of up to 2500 vertices, we replicate their results and propose an alternative approach using SAT and CP-SAT solvers. Our findings show that SAT solvers outperform MIP solvers, allowing us to efficiently solve random simple polygons without holes of up to 30,000 vertices and improve performance on random simple polygons with holes. 

Additionally, we introduce formulations for the Conflict-free Chromatic Art Gallery problem. In this variation of the problem, rather than requiring all guards covering a point to have different colors, we only ask for at least one guard to have a unique color among them. This problem has not been practically considered before and turned out to be a more challenging variation. While we make progress in solving smaller instances of this problem, further investigation is needed for larger polygon sizes.
