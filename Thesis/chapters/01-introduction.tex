%!TeX root=../thesis.tex
\chapter{Introduction}
The Chromatic Art Gallery Problem (CAGP) was first introduced by Erickson and LaValle in 2010 ~\cite{erickson2010chromatic}. The problem involves finding a guard set that covers the entirety of a given polygon P, requiring the smallest number of colors to color the guards such that the visibility regions of guards with the same color do not overlap. This minimum number of colors is called the ``chromatic guard number'' $\chi_G(P)$. Another closely related problem is the Conflict-free Chromatic Art Gallery Problem (CFCAGP), where the goal is for each point in the polygon to have at least one guard with a distinct color among the guards covering that point ~\cite{bartschi2014conflict}. For both problems, finding a guard set that satisfies the constraints has been established to be NP-hard even when limiting oneself to the set of candidate guards being the vertices of the polygon ~\cite{fekete2014chromatic}~\cite{erickson2011many}~\cite{iwamoto2022vertex}. In this context, we will explore finding exact solutions using MIP, SAT, and CP-SAT solvers.

\section{Results}
Both for the Chromatic Art Gallery Problem and its conflict-free variation, we show that SAT solvers are highly efficient compared to MIP solvers. For the CAGP, they allow us to solve random simple polygons without holes of up to 30,000 vertices as well as most random simple polygons of up to 1000 vertices with 100 holes. In the case of random simple polygons without holes, they provide faster runtimes than MIP for all tested instances. In the case of random simple polygons with holes, SAT is faster for most instances and overall can solve more of the test instances within a 600-second time limit.
For the CFAGP, solving even small instances turned out to be much harder. Using SAT solvers, we are still able to solve random simple polygons without holes, 

\section{Related Work}
The complexity of the chromatic art gallery problem has been the subject of several papers covering different classes of polygons.
For a polygon with holes, it is NP-hard to decide if a fixed number $k \geq 2$ of colors is sufficient ~\cite{fekete2014complexity}. Similarly, computing the minimum number of colors needed to cover a simple polygon is NP-hard for $\Theta(n)$ colors ~\cite{fekete2014complexity}, and the same holds when limiting ourselves to arbitrary guard positions ~\cite{fekete2014chromatic}. When it comes to covering an orthogonal polygon, deciding whether a fixed number $k \geq 2$ of colors is sufficient is also NP-hard ~\cite{hoorfar2021np}.\par\noindent
However, given a polygon $P$ and a guard set $S$ of $P$, it is possible to compute an optimal coloring of $S$ so that no two members of $S$ with the same color have overlapping visibility regions in polynomial time ~\cite{erickson2011many}. For a simple polygon $P$ and a discrete set of candidate guard locations, there exists a polynomial time algorithm to compute an optimal $2$-colorable guard set considering various objectives ~\cite{fekete2014chromatic}. Additionally, for a simple polygon $P$ and a discrete set of candidate guard locations, there is an $\mathcal{O}(\log (\chi_G(P)))$-approximation algorithm that runs in polynomial time ~\cite{fekete2014chromatic}. There is also a $6$-approximation algorithm for simple orthogonal polygons with linear time and space complexity, as well as an exact algorithm for histogram polygons that runs in linear time \cite{hoorfar2021np}.\par\noindent
Since the problem remains NP-hard for most polygon classes, one might want to find lower and upper bounds on the problem.\par\noindent
In the following, we consider a few lower bounds.
For $k \geq 3$ exists a polygon $P_k$ with $4k$ vertices and $\chi_G(P_k) \geq k$ ~\cite{erickson2012art}.
For $k$ exists a polygon $P_k$ with $3k$ vertices and $\chi_G(P_k) \geq k/2$ ~\cite{bartschi2011coloring}.
For $k \geq 3$ exists a strictly monotone polygon $M_k$ with $3k^2$ vertices and $\chi_G(M_k) \geq k$ ~\cite{erickson2012art}.
For $k \geq 3$ and odd exists a monotone orthogonal polygon $R_k$ with $4k^2 + 10k + 10$ vertices and $\chi_G(R_k) \geq k$ ~\cite{erickson2012art}.
For $k$ exists a monotone orthogonal polygon $R_k$ with $4k^2$ vertices and $\chi_G(R_k) \geq k/4$ ~\cite{bartschi2011coloring}.\par\noindent
Next, we look at a few upper bounds.
For any monotone, orthogonal or general polygon $P_n$ with n vertices $\chi_G(P_n) = \mathcal{O}(n)$ ~\cite{bartschi2011coloring}.
For any spiral polygon $P_{spi}$ $\chi_G(P_{spi}) \leq 2$ ~\cite{erickson2012art}.
For any staircase polygon $P_{sta}$: $\chi_G(P_{sta}) \leq 3$ ~\cite{erickson2012art}.\par\noindent
The only paper that has addressed practically solving the chromatic art gallery problem so far is by Zambon et al.~\cite{zambon2014exact}. The paper proposes a MIP formulation of the problem and provides practical results on it. The MIP solver used in this thesis is based on the approach presented in that paper.\par\noindent
Now looking at the conflict-free chromatic art gallery problem, there has only been one paper considering its complexity so far.
For a polygon with holes, determining whether there exists a conflict-free chromatic vertex guard set with $2$ colors is NP-hard ~\cite{iwamoto2022vertex}.\par\noindent
As for lower and upper bounds, the following upper bounds were discovered.
For any monotone polygon $P_n$ with n vertices $\chi_{cfG}(P_n) = \mathcal{O}(\log n)$ ~\cite{bartschi2011coloring}.
For any orthogonal polygon $P_n$ with n vertices $\chi_{cfG}(P_n) = \mathcal{O}(\log n)$ ~\cite{bartschi2011coloring}.
For any polygon $P_n$ with n vertices $\chi_{cfG}(P_n) = \mathcal{O}((\log n)^2)$ ~\cite{bartschi2011coloring}.
As for practically solving the Conflict-free Chromatic Art Gallery Problem, there have been no publications as of date and this thesis provides a first attempt at doing so.

% \textbf{Exact algorithm for the DCAGP (Discrete Chromatic Art Gallery Problem):} \cite{zambon2014exact}
% \begin{enumerate}
%     \item Determine guard set G
%     \item Determine witness set W via placing a witness in each shadow AVP
%     \item Build Graph G$_T$ for every guard/witness a vertex and an edge whenever a guard covers a witness or two guards are conflicting
%     \item Use Upper Bound heuristic using independent sets to find a feasible K
%     \item Solve the DCAGP on G$_T$ using K:
%     \begin{enumerate}
%         \item[-] Boolean variables for K colors and for whether a color is assigned to a guard in the optimal solution
%         \item[-] minimize the number of colors used
%         \item[-] add witness covering constraints
%         \item[-] add at most one color per guard constraint (performance)
%         \item[-] add symmetry breaker constraints (performance)
%         \item[-] lazily add clique edge cover constraints (instead of edge constraints for conflicting guards)
%     \end{enumerate}
% \end{enumerate}

\section{Overview}
In the Preliminaries section, we will briefly introduce the terms used throughout this thesis. Afterward in the Theoretical Results section, we prove a few theorems that are necessary for a correct implementation of our problem formulations. In the following two sections, we provide MIP and SAT formulations for the Chromatic Art Gallery Problem and Conflict-free Chromatic Art Gallery Problem respectively. The Implementation Details section deals with the methods used to process our geometric input allowing us to convert it into MIP constraints and SAT clauses. In the Experiments section, we will benchmark our implementations on random simple polygons with and without holes. Lastly, we give a conclusion on our experimental findings and provide open problems for future research.