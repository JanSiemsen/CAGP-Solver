%!TeX root=../thesis.tex
\chapter{Introduction}
The chromatic art gallery problem:
\begin{enumerate}
    \item[-] first introduced by L. H. Erickson and S. M. LaValle in 2010 \cite{erickson2010chromatic}
    \item[-] given a polygon $P$ find a guard set that covers the entirety of $P$ and requires the smallest number of colors to color the guards in a way that the visibility regions of two guards with the same color do not overlap
    \item[-] the number is called \emph{chromatic guard number} $\chi_G(P)$ 
\end{enumerate}

\section{Results}
You can write down your main result here.

\section{Related Work}
\textbf{The chromatic art gallery problem:} \\
\textbf{Complexity:}
\begin{enumerate}
    \item[-] given a polygon P and guard set S of P, we can compute an optimal coloring of S so that no two members of S with the same color have overlapping visibility regions in polynomial time \cite{erickson2011many}
    \item[-] for covering a polygon with holes it is NP-hard to decide whether any fixed number $k \geq 2$ of colors suffices \cite{fekete2014complexity}
    \item[-] for covering a simple polygon to compute the minimum number of colors needed is NP-hard for $\Theta(n)$ colors \cite{fekete2014complexity},
    the same stands when we limit ourselves to arbitrary guard positions \cite{fekete2014chromatic} 
    \item[-] for a simple polygon P given a discrete set of candidate guards locations, there is an $O(\log (\chi_G(P)))$-approximation algorithm in polynomial time \cite{fekete2014chromatic}
    \item[-] for a simple polygon P given a discrete set of candidate guard locations, there is a polynomial time algorithm to compute an optimal 2-colorable guard set regarding various objectives \cite{fekete2014chromatic} 
    \item[-] for covering an orthogonal Polygon it is NP-hard to decide whether any fixed number $k \geq 2$ of colors suffices \cite{hoorfar2021np}
    \item[-] there is a 6-approximation algorithm for simple orthogonal polygons with linear time and space complexity \cite{hoorfar2021np}
    \item[-] there is an exact algorithm for histograms polygons in linear time \cite{hoorfar2021np}
\end{enumerate}
\textbf{Lower bounds:}
\begin{enumerate}
    \item[-] for $k \geq 3$ exists a polygon $P_k$ with $4k$ vertices and $\chi_G(P_k) \geq k$ \cite{erickson2012art}
    \item[-] for $k$ exists a polygon $P_k$ with $3k$ vertices and $\chi_G(P_k) \geq k/2$ \cite{bartschi2011coloring}
    \item[-] for $k \geq 3$ exists a strictly monotone polygon $M_k$ with $3k^2$ vertices and $\chi_G(M_k) \geq k$ \cite{erickson2012art}
    \item[-] for $k \geq 3$ and odd exists a monotone orthogonal polygon $R_k$ with $4k^2 + 10k + 10$ vertices and $\chi_G(R_k) \geq k$ \cite{erickson2012art}
    \item[-] for $k$ exists a monotone orthogonal polygon $R_k$ with $4k^2$ vertices and $\chi_G(R_k) \geq k/4$ \cite{bartschi2011coloring}
\end{enumerate}
\textbf{Upper bounds:}
\begin{enumerate}
    \item[-] for any monotone, orthogonal or general polygon $P_n$ with n vertices:  $\chi_G(P_n) = O(n)$ \cite{bartschi2011coloring}
    \item[-] for any spiral polygon $P_{spi}$: $\chi_G(P_{spi}) \leq 2$ \cite{erickson2012art}
    \item[-] for any staircase polygon $P_{sta}$: $\chi_G(P_{sta}) \leq 3$ \cite{erickson2012art}
\end{enumerate}
\textbf{The conflict-free chromatic art gallery problem:} \\
\textbf{Upper bounds:}
\begin{enumerate}
    \item[-] for any monotone polygon $P_n$ with n vertices:  $\chi_{cfG}(P_n) = O(\log n)$ \cite{bartschi2011coloring}
    \item[-] for any orthogonal polygon $P_n$ with n vertices:  $\chi_{cfG}(P_n) = O(\log n)$ \cite{bartschi2011coloring}
    \item[-] for any polygon $P_n$ with n vertices:  $\chi_{cfG}(P_n) = O((\log n)^2)$ \cite{bartschi2011coloring}
\end{enumerate}
\textbf{Exact algorithm for the DCAGP (Discrete Chromatic Art Gallery Problem):} \cite{zambon2014exact}
\begin{enumerate}
    \item Determine guard set G
    \item Determine witness set W via placing a witness in each shadow AVP
    \item Build Graph G$_T$ for every guard/witness a vertex and an edge whenever a guard covers a witness or two guards are conflicting
    \item Use Upper Bound heuristic using independent sets to find a feasible K
    \item Solve the DCAGP on G$_T$ using K:
    \begin{enumerate}
        \item[-] Boolean variables for K colors and for whether a color is assigned to a guard in the optimal solution
        \item[-] minimize number of colors used
        \item[-] add witness covering constraints
        \item[-] add at most one color per guard constraint (performance)
        \item[-] add symmetry breaker constraints (performance)
        \item[-] lazily add edge clique cover constraints (instead of edge constraints for conflicting guards)
    \end{enumerate}
\end{enumerate}

\section{Overview}
If needed, write down where to find everything.